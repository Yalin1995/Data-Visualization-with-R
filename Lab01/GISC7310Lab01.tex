\PassOptionsToPackage{unicode=true}{hyperref} % options for packages loaded elsewhere
\PassOptionsToPackage{hyphens}{url}
%
\documentclass[
]{article}
\usepackage{lmodern}
\usepackage{amssymb,amsmath}
\usepackage{ifxetex,ifluatex}
\ifnum 0\ifxetex 1\fi\ifluatex 1\fi=0 % if pdftex
  \usepackage[T1]{fontenc}
  \usepackage[utf8]{inputenc}
  \usepackage{textcomp} % provides euro and other symbols
\else % if luatex or xelatex
  \usepackage{unicode-math}
  \defaultfontfeatures{Scale=MatchLowercase}
  \defaultfontfeatures[\rmfamily]{Ligatures=TeX,Scale=1}
\fi
% use upquote if available, for straight quotes in verbatim environments
\IfFileExists{upquote.sty}{\usepackage{upquote}}{}
\IfFileExists{microtype.sty}{% use microtype if available
  \usepackage[]{microtype}
  \UseMicrotypeSet[protrusion]{basicmath} % disable protrusion for tt fonts
}{}
\makeatletter
\@ifundefined{KOMAClassName}{% if non-KOMA class
  \IfFileExists{parskip.sty}{%
    \usepackage{parskip}
  }{% else
    \setlength{\parindent}{0pt}
    \setlength{\parskip}{6pt plus 2pt minus 1pt}}
}{% if KOMA class
  \KOMAoptions{parskip=half}}
\makeatother
\usepackage{xcolor}
\IfFileExists{xurl.sty}{\usepackage{xurl}}{} % add URL line breaks if available
\IfFileExists{bookmark.sty}{\usepackage{bookmark}}{\usepackage{hyperref}}
\hypersetup{
  pdftitle={Lab01: Data Transformations, Bivariate Regression Analysis, Numerical Integration \& Distributions},
  pdfauthor={Yalin Yang},
  pdfborder={0 0 0},
  breaklinks=true}
\urlstyle{same}  % don't use monospace font for urls
\usepackage[margin=1in]{geometry}
\usepackage{color}
\usepackage{fancyvrb}
\newcommand{\VerbBar}{|}
\newcommand{\VERB}{\Verb[commandchars=\\\{\}]}
\DefineVerbatimEnvironment{Highlighting}{Verbatim}{commandchars=\\\{\}}
% Add ',fontsize=\small' for more characters per line
\usepackage{framed}
\definecolor{shadecolor}{RGB}{248,248,248}
\newenvironment{Shaded}{\begin{snugshade}}{\end{snugshade}}
\newcommand{\AlertTok}[1]{\textcolor[rgb]{0.94,0.16,0.16}{#1}}
\newcommand{\AnnotationTok}[1]{\textcolor[rgb]{0.56,0.35,0.01}{\textbf{\textit{#1}}}}
\newcommand{\AttributeTok}[1]{\textcolor[rgb]{0.77,0.63,0.00}{#1}}
\newcommand{\BaseNTok}[1]{\textcolor[rgb]{0.00,0.00,0.81}{#1}}
\newcommand{\BuiltInTok}[1]{#1}
\newcommand{\CharTok}[1]{\textcolor[rgb]{0.31,0.60,0.02}{#1}}
\newcommand{\CommentTok}[1]{\textcolor[rgb]{0.56,0.35,0.01}{\textit{#1}}}
\newcommand{\CommentVarTok}[1]{\textcolor[rgb]{0.56,0.35,0.01}{\textbf{\textit{#1}}}}
\newcommand{\ConstantTok}[1]{\textcolor[rgb]{0.00,0.00,0.00}{#1}}
\newcommand{\ControlFlowTok}[1]{\textcolor[rgb]{0.13,0.29,0.53}{\textbf{#1}}}
\newcommand{\DataTypeTok}[1]{\textcolor[rgb]{0.13,0.29,0.53}{#1}}
\newcommand{\DecValTok}[1]{\textcolor[rgb]{0.00,0.00,0.81}{#1}}
\newcommand{\DocumentationTok}[1]{\textcolor[rgb]{0.56,0.35,0.01}{\textbf{\textit{#1}}}}
\newcommand{\ErrorTok}[1]{\textcolor[rgb]{0.64,0.00,0.00}{\textbf{#1}}}
\newcommand{\ExtensionTok}[1]{#1}
\newcommand{\FloatTok}[1]{\textcolor[rgb]{0.00,0.00,0.81}{#1}}
\newcommand{\FunctionTok}[1]{\textcolor[rgb]{0.00,0.00,0.00}{#1}}
\newcommand{\ImportTok}[1]{#1}
\newcommand{\InformationTok}[1]{\textcolor[rgb]{0.56,0.35,0.01}{\textbf{\textit{#1}}}}
\newcommand{\KeywordTok}[1]{\textcolor[rgb]{0.13,0.29,0.53}{\textbf{#1}}}
\newcommand{\NormalTok}[1]{#1}
\newcommand{\OperatorTok}[1]{\textcolor[rgb]{0.81,0.36,0.00}{\textbf{#1}}}
\newcommand{\OtherTok}[1]{\textcolor[rgb]{0.56,0.35,0.01}{#1}}
\newcommand{\PreprocessorTok}[1]{\textcolor[rgb]{0.56,0.35,0.01}{\textit{#1}}}
\newcommand{\RegionMarkerTok}[1]{#1}
\newcommand{\SpecialCharTok}[1]{\textcolor[rgb]{0.00,0.00,0.00}{#1}}
\newcommand{\SpecialStringTok}[1]{\textcolor[rgb]{0.31,0.60,0.02}{#1}}
\newcommand{\StringTok}[1]{\textcolor[rgb]{0.31,0.60,0.02}{#1}}
\newcommand{\VariableTok}[1]{\textcolor[rgb]{0.00,0.00,0.00}{#1}}
\newcommand{\VerbatimStringTok}[1]{\textcolor[rgb]{0.31,0.60,0.02}{#1}}
\newcommand{\WarningTok}[1]{\textcolor[rgb]{0.56,0.35,0.01}{\textbf{\textit{#1}}}}
\usepackage{graphicx,grffile}
\makeatletter
\def\maxwidth{\ifdim\Gin@nat@width>\linewidth\linewidth\else\Gin@nat@width\fi}
\def\maxheight{\ifdim\Gin@nat@height>\textheight\textheight\else\Gin@nat@height\fi}
\makeatother
% Scale images if necessary, so that they will not overflow the page
% margins by default, and it is still possible to overwrite the defaults
% using explicit options in \includegraphics[width, height, ...]{}
\setkeys{Gin}{width=\maxwidth,height=\maxheight,keepaspectratio}
\setlength{\emergencystretch}{3em}  % prevent overfull lines
\providecommand{\tightlist}{%
  \setlength{\itemsep}{0pt}\setlength{\parskip}{0pt}}
\setcounter{secnumdepth}{-2}
% Redefines (sub)paragraphs to behave more like sections
\ifx\paragraph\undefined\else
  \let\oldparagraph\paragraph
  \renewcommand{\paragraph}[1]{\oldparagraph{#1}\mbox{}}
\fi
\ifx\subparagraph\undefined\else
  \let\oldsubparagraph\subparagraph
  \renewcommand{\subparagraph}[1]{\oldsubparagraph{#1}\mbox{}}
\fi

% set default figure placement to htbp
\makeatletter
\def\fps@figure{htbp}
\makeatother


\title{Lab01: Data Transformations, Bivariate Regression Analysis, Numerical
Integration \& Distributions}
\author{Yalin Yang}
\date{2020-05-03}

\begin{document}
\maketitle

{
\setcounter{tocdepth}{3}
\tableofcontents
}
\hypertarget{task-1}{%
\section{Task 1}\label{task-1}}

\textbf{Univariate Variable Exploration and Transformations {[}2
points{]}}

Use the CPS1985 dataset in the library AER to explore the distribution
of the respondents' wage.

\hypertarget{task-1.1}{%
\subsection{Task 1.1}\label{task-1.1}}

Find the best \(\lambda_{best}\)-value for the Box-Cox transformation
(see summary(car::powerTransform(varName))). Could the
log-transformation (i.e., \(\lambda=0.0\)) instead of \(\lambda^best\)
be used? Justify your answer. {[}0.5 points{]}

\begin{Shaded}
\begin{Highlighting}[]
\KeywordTok{library}\NormalTok{(AER)}
\KeywordTok{data}\NormalTok{(}\StringTok{"CPS1985"}\NormalTok{)}
\KeywordTok{symbox}\NormalTok{(}\OperatorTok{~}\NormalTok{wage, }\DataTypeTok{data=}\NormalTok{CPS1985)}
\end{Highlighting}
\end{Shaded}

\includegraphics{GISC7310Lab01_files/figure-latex/unnamed-chunk-1-1.pdf}

\begin{Shaded}
\begin{Highlighting}[]
\KeywordTok{summary}\NormalTok{(}\KeywordTok{powerTransform}\NormalTok{(}\KeywordTok{lm}\NormalTok{(wage}\OperatorTok{~}\DecValTok{1}\NormalTok{, }\DataTypeTok{data=}\NormalTok{CPS1985)))}
\end{Highlighting}
\end{Shaded}

\begin{verbatim}
## bcPower Transformation to Normality 
##    Est Power Rounded Pwr Wald Lwr Bnd Wald Upr Bnd
## Y1   -0.0658           0      -0.1997        0.068
## 
## Likelihood ratio test that transformation parameter is equal to 0
##  (log transformation)
##                             LRT df    pval
## LR test, lambda = (0) 0.9245772  1 0.33628
## 
## Likelihood ratio test that no transformation is needed
##                            LRT df       pval
## LR test, lambda = (1) 232.5055  1 < 2.22e-16
\end{verbatim}

The estimated transformation power (\(\lambda = -0.0658\)), which is
very close to 0. The p-value of the likelihood ratio test (H0: \lambda =
0 ) is larger than 0.05, so we fail to reject the null hypothesis.
Instead of using (\(\lambda = -0.0658\)), the log-transformation should
be used in this case.

\hypertarget{task-1.2}{%
\subsection{Task 1.2}\label{task-1.2}}

For the untransformed (\(\lambda = 1\)), optimal
(\(\lambda = \lambda𝑏𝑒𝑠𝑡\) ) and over-adjusted (\lambda = -1 ) Box-Cox
transformed wage variable repeat the following tasks and comparatively
interpret the results:

{[}a{]} Draw properly constructed histograms of all three distributions
and discuss their properties

\begin{Shaded}
\begin{Highlighting}[]
\KeywordTok{par}\NormalTok{(}\DataTypeTok{mfrow =} \KeywordTok{c}\NormalTok{(}\DecValTok{1}\NormalTok{,}\DecValTok{3}\NormalTok{))}
\KeywordTok{hist}\NormalTok{(car}\OperatorTok{::}\KeywordTok{bcPower}\NormalTok{(CPS1985}\OperatorTok{$}\NormalTok{wage, }\DataTypeTok{lambda=}\DecValTok{1}\NormalTok{),}\DataTypeTok{breaks =} \DecValTok{12}\NormalTok{,}\DataTypeTok{main =} \StringTok{'lambda = 1'}\NormalTok{,}\DataTypeTok{xlab =} \StringTok{'x'}\NormalTok{)}
\KeywordTok{hist}\NormalTok{(car}\OperatorTok{::}\KeywordTok{bcPower}\NormalTok{(CPS1985}\OperatorTok{$}\NormalTok{wage, }\DataTypeTok{lambda=} \FloatTok{-0.0658}\NormalTok{ ),}\DataTypeTok{breaks =} \DecValTok{12}\NormalTok{,}\DataTypeTok{main =} \StringTok{'lambda = -0.0658'}\NormalTok{,}\DataTypeTok{xlab =} \StringTok{'x'}\NormalTok{)}
\KeywordTok{hist}\NormalTok{(car}\OperatorTok{::}\KeywordTok{bcPower}\NormalTok{(CPS1985}\OperatorTok{$}\NormalTok{wage, }\DataTypeTok{lambda=} \FloatTok{-1.5}\NormalTok{ ),}\DataTypeTok{breaks =} \DecValTok{12}\NormalTok{,}\DataTypeTok{main =} \StringTok{'lambda = -1.5'}\NormalTok{,}\DataTypeTok{xlab =} \StringTok{'x'}\NormalTok{)}
\end{Highlighting}
\end{Shaded}

\includegraphics{GISC7310Lab01_files/figure-latex/unnamed-chunk-3-1.pdf}
\textbf{\(\lambda=1\) are positively skewed, \(\lambda=-0.068\) are
close to the normal distribution with little positive skewness.
\(\lambda=-1.5\) are highly negatively skewed.}

{[}b{]} evaluate their sknewness

\begin{Shaded}
\begin{Highlighting}[]
\NormalTok{skew1 <-}\StringTok{ }\NormalTok{e1071}\OperatorTok{::}\KeywordTok{skewness}\NormalTok{(car}\OperatorTok{::}\KeywordTok{bcPower}\NormalTok{(CPS1985}\OperatorTok{$}\NormalTok{wage, }\DataTypeTok{lambda=}\DecValTok{1}\NormalTok{))}
\NormalTok{skew2 <-}\StringTok{ }\NormalTok{e1071}\OperatorTok{::}\KeywordTok{skewness}\NormalTok{(car}\OperatorTok{::}\KeywordTok{bcPower}\NormalTok{(CPS1985}\OperatorTok{$}\NormalTok{wage, }\DataTypeTok{lambda=}\OperatorTok{-}\FloatTok{0.0658}\NormalTok{))}
\NormalTok{skew3 <-}\StringTok{ }\NormalTok{e1071}\OperatorTok{::}\KeywordTok{skewness}\NormalTok{(car}\OperatorTok{::}\KeywordTok{bcPower}\NormalTok{(CPS1985}\OperatorTok{$}\NormalTok{wage, }\DataTypeTok{lambda=}\OperatorTok{-}\FloatTok{1.5}\NormalTok{))}
\KeywordTok{data.frame}\NormalTok{(}\DataTypeTok{skews =} \KeywordTok{c}\NormalTok{(skew1,skew2,skew3))}
\end{Highlighting}
\end{Shaded}

\begin{verbatim}
##          skews
## 1  1.687762093
## 2  0.001027028
## 3 -7.310307169
\end{verbatim}

{[}c{]} test whether the variables are approximately normal distributed
(see \texttt{shapiro.test(\ )}).

Address also the questions: Which transformed variable comes closest to
the normal distribution? Is the transformation with \(\lambda = -1.5\)
over-compensating the inherent positive skewness in the wage
variable?{[}1.5 points{]}

\begin{Shaded}
\begin{Highlighting}[]
\KeywordTok{shapiro.test}\NormalTok{(car}\OperatorTok{::}\KeywordTok{bcPower}\NormalTok{(CPS1985}\OperatorTok{$}\NormalTok{wage, }\DataTypeTok{lambda=}\DecValTok{1}\NormalTok{))}
\end{Highlighting}
\end{Shaded}

\begin{verbatim}
## 
##  Shapiro-Wilk normality test
## 
## data:  car::bcPower(CPS1985$wage, lambda = 1)
## W = 0.8673, p-value < 2.2e-16
\end{verbatim}

\begin{Shaded}
\begin{Highlighting}[]
\KeywordTok{shapiro.test}\NormalTok{(car}\OperatorTok{::}\KeywordTok{bcPower}\NormalTok{(CPS1985}\OperatorTok{$}\NormalTok{wage, }\DataTypeTok{lambda=}\OperatorTok{-}\FloatTok{0.0658}\NormalTok{))}
\end{Highlighting}
\end{Shaded}

\begin{verbatim}
## 
##  Shapiro-Wilk normality test
## 
## data:  car::bcPower(CPS1985$wage, lambda = -0.0658)
## W = 0.98923, p-value = 0.000586
\end{verbatim}

\begin{Shaded}
\begin{Highlighting}[]
\KeywordTok{shapiro.test}\NormalTok{(car}\OperatorTok{::}\KeywordTok{bcPower}\NormalTok{(CPS1985}\OperatorTok{$}\NormalTok{wage, }\DataTypeTok{lambda=}\OperatorTok{-}\FloatTok{1.5}\NormalTok{))}
\end{Highlighting}
\end{Shaded}

\begin{verbatim}
## 
##  Shapiro-Wilk normality test
## 
## data:  car::bcPower(CPS1985$wage, lambda = -1.5)
## W = 0.59422, p-value < 2.2e-16
\end{verbatim}

The untransformed wage variable has positive skewness with an outlier
\$44500. The optimal transformation makes the transformed distribution
almost symmetric with a tiny skewness value. However, the positive
skewness is over-compensated when using (\(\lambda = -1.5\)). This leads
to substantial negative skewness

The p-values of Shapiro-Wilk normality tests with
\(H_0 : X\)\textasciitilde{}\(N(\hat{\mu},\hat{\sigma^2})\) for the
properly Box-Cox transformed data is much smaller than 0.05, therefore
transformed wage still deviates from the normal distribution. However,
this p-value is the largest one among all three scenarios. Therefore, we
can conclude the optimal transformed variable becomes closest to the
normal distribution.

\hypertarget{task-2}{%
\section{Task 2}\label{task-2}}

\textbf{Explore the function powerTransform to achieve a multivariate
normal distribution {[}1 point{]}}

\hypertarget{task-2.1}{%
\subsection{Task 2.1}\label{task-2.1}}

\textbf{Simultaneously} estimate the optimal set of Box-Cox
transformation parameters for all variables so that the set transformed
variables becomes approximately multivariate normal distributed. Report
your code to do the estimation. {[}0.5 points{]}

\begin{Shaded}
\begin{Highlighting}[]
\NormalTok{Concord <-}\StringTok{ }\NormalTok{foreign}\OperatorTok{::}\KeywordTok{read.spss}\NormalTok{(}\StringTok{"Concord1.sav "}\NormalTok{, }\DataTypeTok{to.data.frame=} \OtherTok{TRUE}\NormalTok{)}
\NormalTok{Concord <-}\StringTok{ }\KeywordTok{na.omit}\NormalTok{(Concord)}
\KeywordTok{summary}\NormalTok{(lambda <-}\StringTok{ }\KeywordTok{powerTransform}\NormalTok{(}\KeywordTok{lm}\NormalTok{(}\KeywordTok{cbind}\NormalTok{(water79,water80,water81)}\OperatorTok{~}\DecValTok{1}\NormalTok{, }\DataTypeTok{data=}\NormalTok{Concord)))}
\end{Highlighting}
\end{Shaded}

\begin{verbatim}
## bcPower Transformations to Multinormality 
##         Est Power Rounded Pwr Wald Lwr Bnd Wald Upr Bnd
## water79    0.1651        0.17       0.0824       0.2478
## water80    0.1782        0.18       0.1005       0.2558
## water81    0.2717        0.33       0.2022       0.3412
## 
## Likelihood ratio test that transformation parameters are equal to 0
##  (all log transformations)
##                                LRT df       pval
## LR test, lambda = (0 0 0) 77.71615  3 < 2.22e-16
## 
## Likelihood ratio test that no transformations are needed
##                                LRT df       pval
## LR test, lambda = (1 1 1) 716.6848  3 < 2.22e-16
\end{verbatim}

p-values for transformation vectors (\(\lambda = 0,0,0\)), i.e., perform
a log-transformation on all variables, and (\(\lambda = 1,1,1\)), i.e.,
leave all variables untransformed, are much smaller than 𝛼 = 0.05 so we
can reject the two null hypotheses assuming these sets of transformation
parameters lead to a multivariate normal distribution for the set of
variables. The estimated transformation vector
(\(\lambda = 0.17, 0.18, 0.33\)) comes the closest to the multivariate
normal distribution.

\hypertarget{task-2.2}{%
\subsection{Task 2.2}\label{task-2.2}}

Show the output and interpret the results. {[}0.5 points{]}

\begin{Shaded}
\begin{Highlighting}[]
\NormalTok{Concord <-}\StringTok{ }\KeywordTok{data.frame}\NormalTok{(Concord,                      }\CommentTok{# add transformed variables to myPower}
                      \KeywordTok{bcPower}\NormalTok{(}\KeywordTok{cbind}\NormalTok{(Concord}\OperatorTok{$}\NormalTok{water79,Concord}\OperatorTok{$}\NormalTok{water80,Concord}\OperatorTok{$}\NormalTok{water81), }\KeywordTok{coef}\NormalTok{(lambda, }\DataTypeTok{round=}\NormalTok{T)))}
\KeywordTok{par}\NormalTok{(}\DataTypeTok{mfrow =} \KeywordTok{c}\NormalTok{(}\DecValTok{1}\NormalTok{,}\DecValTok{3}\NormalTok{))}
\KeywordTok{hist}\NormalTok{(Concord}\OperatorTok{$}\NormalTok{Z1.}\FloatTok{0.17}\NormalTok{,}\DataTypeTok{breaks =} \DecValTok{12}\NormalTok{,}\DataTypeTok{main =} \KeywordTok{paste}\NormalTok{(}\StringTok{'Water79 , Skewness ='}\NormalTok{,}\KeywordTok{round}\NormalTok{(e1071}\OperatorTok{::}\KeywordTok{skewness}\NormalTok{(Concord}\OperatorTok{$}\NormalTok{Z1.}\FloatTok{0.17}\NormalTok{),}\DecValTok{2}\NormalTok{)),}\DataTypeTok{xlab =} \StringTok{'x'}\NormalTok{)}
\KeywordTok{hist}\NormalTok{(Concord}\OperatorTok{$}\NormalTok{Z2.}\FloatTok{0.18}\NormalTok{,}\DataTypeTok{breaks =} \DecValTok{12}\NormalTok{,}\DataTypeTok{main =} \KeywordTok{paste}\NormalTok{(}\StringTok{'Water80 , Skewness ='}\NormalTok{,}\KeywordTok{round}\NormalTok{(e1071}\OperatorTok{::}\KeywordTok{skewness}\NormalTok{(Concord}\OperatorTok{$}\NormalTok{Z2.}\FloatTok{0.18}\NormalTok{),}\DecValTok{2}\NormalTok{)),}\DataTypeTok{xlab =} \StringTok{'x'}\NormalTok{)}
\KeywordTok{hist}\NormalTok{(Concord}\OperatorTok{$}\NormalTok{Z3.}\FloatTok{0.33}\NormalTok{,}\DataTypeTok{breaks =} \DecValTok{12}\NormalTok{,}\DataTypeTok{main =} \KeywordTok{paste}\NormalTok{(}\StringTok{'Water81 , Skewness ='}\NormalTok{,}\KeywordTok{round}\NormalTok{(e1071}\OperatorTok{::}\KeywordTok{skewness}\NormalTok{(Concord}\OperatorTok{$}\NormalTok{Z3.}\FloatTok{0.33}\NormalTok{),}\DecValTok{2}\NormalTok{)),}\DataTypeTok{xlab =} \StringTok{'x'}\NormalTok{)}
\end{Highlighting}
\end{Shaded}

\includegraphics{GISC7310Lab01_files/figure-latex/unnamed-chunk-7-1.pdf}

\hypertarget{task-3}{%
\section{Task 3}\label{task-3}}

\textbf{Confidence Intervals {[}2 points{]}} Continue with the CPS1985
dataset for this task. To simplify things do not perform variable
transformations.

\hypertarget{task-3.1}{%
\subsection{Task 3.1}\label{task-3.1}}

Run a \textbf{bivariate regression} of wage (dependent variable) on
education (independent variable) and \textbf{interpret} the model
estimates. {[}0.5 points{]}

\begin{Shaded}
\begin{Highlighting}[]
\NormalTok{reg01 <-}\StringTok{ }\KeywordTok{lm}\NormalTok{(wage}\OperatorTok{~}\NormalTok{education, }\DataTypeTok{data=}\NormalTok{CPS1985)}
\KeywordTok{summary}\NormalTok{(reg01)}
\end{Highlighting}
\end{Shaded}

\begin{verbatim}
## 
## Call:
## lm(formula = wage ~ education, data = CPS1985)
## 
## Residuals:
##    Min     1Q Median     3Q    Max 
## -7.911 -3.260 -0.760  2.240 34.740 
## 
## Coefficients:
##             Estimate Std. Error t value Pr(>|t|)    
## (Intercept) -0.74598    1.04545  -0.714    0.476    
## education    0.75046    0.07873   9.532   <2e-16 ***
## ---
## Signif. codes:  0 '***' 0.001 '**' 0.01 '*' 0.05 '.' 0.1 ' ' 1
## 
## Residual standard error: 4.754 on 532 degrees of freedom
## Multiple R-squared:  0.1459, Adjusted R-squared:  0.1443 
## F-statistic: 90.85 on 1 and 532 DF,  p-value: < 2.2e-16
\end{verbatim}

The \(R^2\) is only 0.1459 which means only about 14.6\% of the
variation in the dependent variable wage can be explained by the
independent variable education. The statistically significant slope (H0:
𝛽1 = 0 ) is positive meaning with each additional year of education the
income will increase the hourly wages by \$0.75. While the intercept is
not statistically different from zero, it should be keep in the model
because {[}a{]} there is no logical reason why a person without
education may have zero wages, and {[}b{]} because otherwise some
statistics of the OLS model, such as the 𝑅2 lose their properties.

\hypertarget{task-3.2}{%
\subsection{Task 3.2}\label{task-3.2}}

Calculate the 99 \% confidence intervals around the estimated regression
parameters. Can you draw the same conclusion as you did using the
\textbf{t-test} in the summary output of task 3.1? {[}0.5 points{]}

\begin{Shaded}
\begin{Highlighting}[]
\KeywordTok{cbind}\NormalTok{(}\StringTok{"Coef"}\NormalTok{=}\KeywordTok{coef}\NormalTok{(reg01), }\KeywordTok{confint}\NormalTok{(reg01, }\DataTypeTok{level=}\FloatTok{0.99}\NormalTok{))}
\end{Highlighting}
\end{Shaded}

\begin{verbatim}
##                   Coef     0.5 %    99.5 %
## (Intercept) -0.7459797 -3.448585 1.9566260
## education    0.7504608  0.546926 0.9539955
\end{verbatim}

The t-test investigates the null hypotheses that the estimated
regression parameters are zero. That is, H0: 𝛽0 = 0 for the intercept
and H0: 𝛽1 = 0 for the slope. As long as the 1 - 𝛼 confidence intervals
cover the values under the null hypothesis, that is 𝛽0 = 0 and 𝛽1 = 0,
the null hypothesis cannot be rejected with an error probability of 𝛼.

\hypertarget{task-3.3}{%
\subsection{Task 3.3}\label{task-3.3}}

\textbf{Scatterplot} both variables and add the predicted regression
line as well as the lower and upper 90\% confidence interval lines
around the \textbf{point predictions}.(i.e., prediction interval in
Hamilton and interval=``prediction'' in the predict function).

\begin{Shaded}
\begin{Highlighting}[]
\KeywordTok{library}\NormalTok{(ggplot2)}
\KeywordTok{attach}\NormalTok{(CPS1985)}
\NormalTok{Pred <-}\StringTok{ }\KeywordTok{predict}\NormalTok{(reg01, }\DataTypeTok{interval=}\StringTok{"prediction"}\NormalTok{, }\DataTypeTok{level =} \FloatTok{0.90}\NormalTok{)}
\NormalTok{CPS1985_new <-}\StringTok{ }\KeywordTok{cbind}\NormalTok{(CPS1985,Pred)}
\NormalTok{q <-}\StringTok{ }\KeywordTok{ggplot}\NormalTok{(CPS1985_new,}\KeywordTok{aes}\NormalTok{(education,wage)) }\OperatorTok{+}\StringTok{ }\KeywordTok{geom_point}\NormalTok{() }\OperatorTok{+}\StringTok{ }
\StringTok{  }\KeywordTok{geom_line}\NormalTok{(}\KeywordTok{aes}\NormalTok{(}\DataTypeTok{y=}\NormalTok{lwr), }\DataTypeTok{color =} \StringTok{"blue"}\NormalTok{, }\DataTypeTok{linetype =} \StringTok{"dashed"}\NormalTok{)}\OperatorTok{+}
\StringTok{  }\KeywordTok{geom_line}\NormalTok{(}\KeywordTok{aes}\NormalTok{(}\DataTypeTok{y=}\NormalTok{upr), }\DataTypeTok{color =} \StringTok{"blue"}\NormalTok{, }\DataTypeTok{linetype =} \StringTok{"dashed"}\NormalTok{)}
\NormalTok{q <-}\StringTok{ }\NormalTok{q }\OperatorTok{+}\StringTok{ }\KeywordTok{geom_vline}\NormalTok{(}\DataTypeTok{xintercept =}\KeywordTok{mean}\NormalTok{(education),}\DataTypeTok{color =} \StringTok{"red"}\NormalTok{) }\OperatorTok{+}\StringTok{ }
\StringTok{  }\KeywordTok{geom_hline}\NormalTok{(}\DataTypeTok{yintercept =}\KeywordTok{mean}\NormalTok{(wage),}\DataTypeTok{color =} \StringTok{"red"}\NormalTok{)}
\NormalTok{(q }\OperatorTok{+}\StringTok{ }\KeywordTok{ggtitle}\NormalTok{(}\StringTok{'CPS1985:Education Level against Wage'}\NormalTok{))}
\end{Highlighting}
\end{Shaded}

\includegraphics{GISC7310Lab01_files/figure-latex/unnamed-chunk-10-1.pdf}

\hypertarget{task-4}{%
\section{Task 4}\label{task-4}}

\textbf{Calibration and Prediction of a Bivariate Regression Model with
Skewed Variables {[}3 points{]}}

The DBASE file CampusCrime.dbf has among other variables the count
variables crime (number of crimes committed on university campuses) and
police (size of the campuses police forces).

\hypertarget{task-4.1}{%
\subsection{Task 4.1}\label{task-4.1}}

Plot \texttt{police} in dependence of \texttt{crime} including their
box-plots along the margins. Is a data transformation advisable?

\begin{Shaded}
\begin{Highlighting}[]
\NormalTok{Crime <-}\StringTok{ }\NormalTok{foreign}\OperatorTok{::}\KeywordTok{read.dbf}\NormalTok{(}\StringTok{"CampusCrime.dbf"}\NormalTok{ )}
\KeywordTok{scatterplot}\NormalTok{(police }\OperatorTok{~}\StringTok{ }\NormalTok{crime , }\DataTypeTok{data =}\NormalTok{ Crime,}\DataTypeTok{pch=}\DecValTok{1}\NormalTok{, }\DataTypeTok{smooth=}\KeywordTok{list}\NormalTok{(}\DataTypeTok{span =} \FloatTok{0.35}\NormalTok{,}\DataTypeTok{lty.smooth=}\DecValTok{1}\NormalTok{, }\DataTypeTok{col.smooth=}\StringTok{"red"}\NormalTok{, }\DataTypeTok{col.var=}\StringTok{"red"}\NormalTok{),}
                  \DataTypeTok{regLine=}\KeywordTok{list}\NormalTok{(}\DataTypeTok{col=}\StringTok{"green"}\NormalTok{))}
\end{Highlighting}
\end{Shaded}

\includegraphics{GISC7310Lab01_files/figure-latex/unnamed-chunk-11-1.pdf}

The dependent variable police and the independent variable crime should
be transformed because both are positively skewed. Furthermore, the
residuals of the linear regression model are also slightly positively
skewed. To make sure the residuals satisfy the assumption of ordinary
least squares, it is advisable that both variables are transformed

\hypertarget{task-4.2}{%
\subsection{Task 4.2}\label{task-4.2}}

Find a proper transformation of both variables in a way that the
independent variable crime is approximately symmetrically distributed
and that the transformation of the dependent variable police leads to
approximately symmetrically distributed regression residuals.

\begin{Shaded}
\begin{Highlighting}[]
\KeywordTok{summary}\NormalTok{(}\KeywordTok{powerTransform}\NormalTok{(}\KeywordTok{lm}\NormalTok{(crime}\OperatorTok{~}\DecValTok{1}\NormalTok{, }\DataTypeTok{data=}\NormalTok{Crime)))}
\end{Highlighting}
\end{Shaded}

\begin{verbatim}
## Warning in model.matrix.default(mt, mf, contrasts): non-list contrasts argument
## ignored
\end{verbatim}

\begin{verbatim}
## bcPower Transformation to Normality 
##    Est Power Rounded Pwr Wald Lwr Bnd Wald Upr Bnd
## Y1    0.0258           0      -0.1444        0.196
## 
## Likelihood ratio test that transformation parameter is equal to 0
##  (log transformation)
##                              LRT df    pval
## LR test, lambda = (0) 0.08843754  1 0.76617
## 
## Likelihood ratio test that no transformation is needed
##                            LRT df       pval
## LR test, lambda = (1) 107.9414  1 < 2.22e-16
\end{verbatim}

\begin{Shaded}
\begin{Highlighting}[]
\KeywordTok{summary}\NormalTok{(}\KeywordTok{powerTransform}\NormalTok{((}\KeywordTok{lm}\NormalTok{(police}\OperatorTok{~}\KeywordTok{log}\NormalTok{(crime), }\DataTypeTok{data=}\NormalTok{Crime))))}
\end{Highlighting}
\end{Shaded}

\begin{verbatim}
## Warning in model.matrix.default(mt, mf, contrasts): non-list contrasts argument
## ignored
\end{verbatim}

\begin{verbatim}
## bcPower Transformation to Normality 
##    Est Power Rounded Pwr Wald Lwr Bnd Wald Upr Bnd
## Y1   -0.0051           0      -0.2474       0.2372
## 
## Likelihood ratio test that transformation parameter is equal to 0
##  (log transformation)
##                               LRT df    pval
## LR test, lambda = (0) 0.001726968  1 0.96685
## 
## Likelihood ratio test that no transformation is needed
##                            LRT df       pval
## LR test, lambda = (1) 63.30872  1 1.7764e-15
\end{verbatim}

The suggested lambda parameters are \(\lambda = 0.0258\) for the
independent variable and \(\lambda = -.0051\) for the regression model
lm(police\textasciitilde{}log(crime), data=crime) so that the residuals
are approximately normal or at least symmetrically distributed.

\hypertarget{task-4.3}{%
\subsection{Task 4.3}\label{task-4.3}}

Test whether a log-transformation (i.e., \(\lambda = 0\)) is appropriate
for the dependent and the independent variables If the
log-transformation is appropriate then use it because their relationship
can now be interpreted in terms of elasticities.

The likelihood ratio tests in task 4.2 suggests that the estimated
lambda coefficients are not significantly different from zero.
Therefore, both crime and police can be log-transformed.

\hypertarget{task-4.4}{%
\subsection{Task 4.4}\label{task-4.4}}

Estimate the model in the transformed system and interpret the
estimates. Also test if the elasticity (i.e., slope parameter) differs
significantly from the neutral elasticity of 1, i.e., H0: 𝛽1 = 1. This
could be done manually by using 𝛽1's standard error from the regression
output or by using the function \texttt{car::linearHypothesis}.

\begin{Shaded}
\begin{Highlighting}[]
\NormalTok{elast.lm <-}\StringTok{ }\KeywordTok{lm}\NormalTok{(}\KeywordTok{log}\NormalTok{(police)}\OperatorTok{~}\KeywordTok{log}\NormalTok{(crime), }\DataTypeTok{data=}\NormalTok{Crime)}
\KeywordTok{summary}\NormalTok{(elast.lm)}
\end{Highlighting}
\end{Shaded}

\begin{verbatim}
## 
## Call:
## lm(formula = log(police) ~ log(crime), data = Crime)
## 
## Residuals:
##     Min      1Q  Median      3Q     Max 
## -1.4893 -0.2910 -0.0575  0.3301  1.3459 
## 
## Coefficients:
##             Estimate Std. Error t value Pr(>|t|)    
## (Intercept)  0.27907    0.23950   1.165    0.247    
## log(crime)   0.46573    0.04326  10.766   <2e-16 ***
## ---
## Signif. codes:  0 '***' 0.001 '**' 0.01 '*' 0.05 '.' 0.1 ' ' 1
## 
## Residual standard error: 0.4793 on 91 degrees of freedom
## Multiple R-squared:  0.5602, Adjusted R-squared:  0.5554 
## F-statistic: 115.9 on 1 and 91 DF,  p-value: < 2.2e-16
\end{verbatim}

Since the bivariate regression model is specified in the log-log form,
the results can be interpreted in terms of elasticity. One percent
change in the number of crimes committed on university campuses will
lead to 0.47 percent change in the size of the campuses police forces.
The meaningful null hypothesis for elasticity is that 1\% in the
independent variable will lead to 1\% change in the dependent variable.

\begin{Shaded}
\begin{Highlighting}[]
\CommentTok{## Test for H0: beta_log(crime)=1}
\NormalTok{slope <-}\StringTok{ }\KeywordTok{coef}\NormalTok{(elast.lm)[}\DecValTok{2}\NormalTok{]}
\NormalTok{se <-}\StringTok{ }\KeywordTok{summary}\NormalTok{(elast.lm)}\OperatorTok{$}\NormalTok{coefficients[}\DecValTok{2}\NormalTok{, }\DecValTok{2}\NormalTok{]}
\NormalTok{df <-}\StringTok{ }\KeywordTok{nrow}\NormalTok{(Crime) }\OperatorTok{-}\StringTok{ }\DecValTok{2}
\NormalTok{(t.value <-}\StringTok{ }\NormalTok{(slope}\DecValTok{-1}\NormalTok{)}\OperatorTok{/}\NormalTok{se) }\CommentTok{# Note E(slope)=1 under H0}
\end{Highlighting}
\end{Shaded}

\begin{verbatim}
## log(crime) 
##  -12.35116
\end{verbatim}

\begin{Shaded}
\begin{Highlighting}[]
\DecValTok{2}\OperatorTok{*}\KeywordTok{pt}\NormalTok{(}\OperatorTok{-}\KeywordTok{abs}\NormalTok{(t.value),}\DataTypeTok{df =}\NormalTok{ df)}
\end{Highlighting}
\end{Shaded}

\begin{verbatim}
##   log(crime) 
## 3.686307e-21
\end{verbatim}

\begin{Shaded}
\begin{Highlighting}[]
\CommentTok{## Alternative approach. Note: F == t.value^2}
\KeywordTok{linearHypothesis}\NormalTok{(elast.lm, }\KeywordTok{c}\NormalTok{(}\StringTok{"log(crime) = 1"}\NormalTok{))}
\end{Highlighting}
\end{Shaded}

\begin{verbatim}
## Linear hypothesis test
## 
## Hypothesis:
## log(crime) = 1
## 
## Model 1: restricted model
## Model 2: log(police) ~ log(crime)
## 
##   Res.Df    RSS Df Sum of Sq      F    Pr(>F)    
## 1     92 55.962                                  
## 2     91 20.910  1    35.052 152.55 < 2.2e-16 ***
## ---
## Signif. codes:  0 '***' 0.001 '**' 0.01 '*' 0.05 '.' 0.1 ' ' 1
\end{verbatim}

The p value is virtually zero for this test, thus we can reject this
null hypothesis H0: 𝛽 = 1. The elasticity 0.46573 of the regression
model is significantly different from unity, which means the relative
change of police is much less elastic than the relative change of crime.
Ultimately it means that the relationship exhibits decreasing rates of
law enforcement allocation. Notice: This is a two-sided test. Thus the
p-value needs to be doubled when you look up the table for only one
tail. The F-statistic of the linearHypothesis( ) function is equal to
the squared t-value.

\hypertarget{task-4.5}{%
\subsection{Task 4.5}\label{task-4.5}}

Perform a prediction in the original data units and plot the
\textbf{median} and \textbf{expectation} curves. Interpret the plot both
in terms of the median and expectation curves.

\begin{Shaded}
\begin{Highlighting}[]
\NormalTok{bcReverseMedian <-}\StringTok{ }\ControlFlowTok{function}\NormalTok{(y, lambda)\{}
\CommentTok{##  }
\CommentTok{## Predicted Median of Reverse Box-Cox Transformation}
\CommentTok{##}
  \ControlFlowTok{if}\NormalTok{ (}\KeywordTok{abs}\NormalTok{(lambda) }\OperatorTok{<}\StringTok{ }\FloatTok{1.0E-6}\NormalTok{)                                         }\CommentTok{#1.0E-6 matches bcPower=0}
\NormalTok{    rev.med <-}\StringTok{ }\KeywordTok{exp}\NormalTok{(y) }\ControlFlowTok{else}
\NormalTok{    rev.med <-}\StringTok{ }\NormalTok{(lambda}\OperatorTok{*}\NormalTok{y}\OperatorTok{+}\DecValTok{1}\NormalTok{)}\OperatorTok{^}\NormalTok{(}\DecValTok{1}\OperatorTok{/}\NormalTok{lambda)}
  \KeywordTok{return}\NormalTok{(rev.med)}
\NormalTok{\} }\CommentTok{# end:revBoxCoxMed}

\NormalTok{bcReverseExpectation <-}\StringTok{ }\ControlFlowTok{function}\NormalTok{(y, lambda, sigma)\{}
\CommentTok{##}
\CommentTok{## Predicted Expectation of Reverse Box-Cox Transformation}
\CommentTok{##}
  \ControlFlowTok{if}\NormalTok{ (}\KeywordTok{abs}\NormalTok{(lambda) }\OperatorTok{<}\StringTok{ }\FloatTok{1.0E-6}\NormalTok{)                                         }\CommentTok{#1.0E-6 matches bcPower=0 }
\NormalTok{    rev.exp <-}\StringTok{ }\KeywordTok{exp}\NormalTok{(y}\FloatTok{+0.5}\OperatorTok{*}\NormalTok{sigma) }\ControlFlowTok{else}
\NormalTok{    rev.exp <-}\StringTok{ }\NormalTok{(lambda}\OperatorTok{*}\NormalTok{y}\OperatorTok{+}\DecValTok{1}\NormalTok{)}\OperatorTok{^}\NormalTok{(}\DecValTok{1}\OperatorTok{/}\NormalTok{lambda)}\OperatorTok{*}\NormalTok{(}\DecValTok{1}\FloatTok{+0.5}\OperatorTok{*}\NormalTok{sigma}\OperatorTok{*}\NormalTok{(}\DecValTok{1}\OperatorTok{-}\NormalTok{lambda)}\OperatorTok{/}\NormalTok{(lambda}\OperatorTok{*}\NormalTok{y}\OperatorTok{+}\DecValTok{1}\NormalTok{)}\OperatorTok{^}\DecValTok{2}\NormalTok{)}
  \KeywordTok{return}\NormalTok{(rev.exp)}
\NormalTok{\} }\CommentTok{# end:bcRevExp}

\NormalTok{plotBoxCox <-}\StringTok{ }\ControlFlowTok{function}\NormalTok{(y, x, ylambda, xlambda)\{ }
\CommentTok{###################################################################}
\CommentTok{## calibrates the model lm(bcPower(y,ylambda)~bcPower(x,xlambda))##}
\CommentTok{## Performs a prediction in the transformed system               ##}
\CommentTok{## Maps predicted values back into the untransformed system      ##}
\CommentTok{###################################################################}
  \KeywordTok{require}\NormalTok{(car)}
  \CommentTok{## Transform both variables}
\NormalTok{  y.bc <-}\StringTok{ }\KeywordTok{bcPower}\NormalTok{(y,ylambda)}
\NormalTok{  x.bc <-}\StringTok{ }\KeywordTok{bcPower}\NormalTok{(x,xlambda)}
  \CommentTok{## Calibrate transformed model and perform prediciton}
\NormalTok{  lm.mod <-}\StringTok{ }\KeywordTok{lm}\NormalTok{(y.bc}\OperatorTok{~}\NormalTok{x.bc)}
\NormalTok{  sigma <-}\StringTok{ }\KeywordTok{sum}\NormalTok{(lm.mod}\OperatorTok{$}\NormalTok{residuals}\OperatorTok{^}\DecValTok{2}\NormalTok{)}\OperatorTok{/}\NormalTok{(}\KeywordTok{length}\NormalTok{(lm.mod}\OperatorTok{$}\NormalTok{residuals)}\OperatorTok{-}\DecValTok{2}\NormalTok{)        }\CommentTok{# estimate residual variance}
\NormalTok{  x.line <-}\StringTok{ }\KeywordTok{data.frame}\NormalTok{(}\DataTypeTok{x.bc=}\KeywordTok{seq}\NormalTok{(}\KeywordTok{min}\NormalTok{(x.bc),}\KeywordTok{max}\NormalTok{(x.bc),}\DataTypeTok{length.out=}\DecValTok{1000}\NormalTok{))  }\CommentTok{# get smooth line}
\NormalTok{  pred.line <-}\StringTok{ }\KeywordTok{predict}\NormalTok{(lm.mod,x.line)                                  }\CommentTok{# predict y along line}

\NormalTok{  y.med.line <-}\StringTok{ }\KeywordTok{bcReverseMedian}\NormalTok{(pred.line,ylambda)                     }\CommentTok{# predicted median}
\NormalTok{  y.exp.line <-}\StringTok{ }\KeywordTok{bcReverseExpectation}\NormalTok{(pred.line,ylambda,sigma)          }\CommentTok{# predicted expectation}
\NormalTok{  x.med.line <-}\StringTok{ }\KeywordTok{bcReverseMedian}\NormalTok{(x.line}\OperatorTok{$}\NormalTok{x.bc,xlambda)                   }\CommentTok{# rescaled independent variable}

  \CommentTok{## Plot data in the original measurement system}
  \KeywordTok{plot}\NormalTok{(x,y,}\DataTypeTok{pch=}\DecValTok{20}\NormalTok{,                                                 }
       \DataTypeTok{xlab=}\KeywordTok{deparse}\NormalTok{(}\KeywordTok{substitute}\NormalTok{(x)), }\DataTypeTok{ylab=}\KeywordTok{deparse}\NormalTok{(}\KeywordTok{substitute}\NormalTok{(y)),   }
       \DataTypeTok{main=}\StringTok{"Predictions after Reverse Box-Cox Transformation"}\NormalTok{)}
  \CommentTok{## Plot prediciton lines}
  \KeywordTok{lines}\NormalTok{(x.med.line,y.med.line,}\DataTypeTok{col=}\StringTok{"green"}\NormalTok{,}\DataTypeTok{lwd=}\DecValTok{2}\NormalTok{)                       }\CommentTok{# conditional median}
  \KeywordTok{lines}\NormalTok{(x.med.line,y.exp.line,}\DataTypeTok{col=}\StringTok{"red"}\NormalTok{,}\DataTypeTok{lwd=}\DecValTok{2}\NormalTok{)                         }\CommentTok{# conditional expectation}
  \KeywordTok{legend}\NormalTok{(}\StringTok{"topright"}\NormalTok{,}\DataTypeTok{title=}\StringTok{"Conditional Predictions"}\NormalTok{,}\DataTypeTok{inset=}\FloatTok{0.01}\NormalTok{,}\DataTypeTok{bg=}\StringTok{"white"}\NormalTok{,}
         \DataTypeTok{legend=}\KeywordTok{c}\NormalTok{(}\StringTok{"Median Method"}\NormalTok{,}\StringTok{"Expectation Method"}\NormalTok{),}\DataTypeTok{col=}\KeywordTok{c}\NormalTok{(}\StringTok{"green"}\NormalTok{,}\StringTok{"red"}\NormalTok{),}\DataTypeTok{lwd=}\DecValTok{2}\NormalTok{)}
\NormalTok{  pred.pts <-}\StringTok{ }\KeywordTok{predict}\NormalTok{(lm.mod)}
\NormalTok{  pred <-}\StringTok{ }\KeywordTok{data.frame}\NormalTok{(}\DataTypeTok{y=}\NormalTok{ y, }\DataTypeTok{y.pred.med=} \KeywordTok{bcReverseMedian}\NormalTok{(pred.pts,ylambda), }
                     \DataTypeTok{y.pred.exp=} \KeywordTok{bcReverseExpectation}\NormalTok{(pred.pts,ylambda,sigma),}\DataTypeTok{x=}\NormalTok{x)}
  \KeywordTok{invisible}\NormalTok{(pred)}
\NormalTok{\} }\CommentTok{#end:plotBoxCox}

\KeywordTok{plotBoxCox}\NormalTok{(Crime}\OperatorTok{$}\NormalTok{police,Crime}\OperatorTok{$}\NormalTok{crime,}\DecValTok{0}\NormalTok{,}\DecValTok{0}\NormalTok{)}
\end{Highlighting}
\end{Shaded}

\includegraphics{GISC7310Lab01_files/figure-latex/unnamed-chunk-18-1.pdf}
For the predictions being mapped back into the original scale, the
expected predicted value is larger than the median predicted value
because mean is larger than median in the positively skewed
distribution. This applies over the full data range of the independent
variable.

\hypertarget{task-5}{%
\section{Task 5}\label{task-5}}

\textbf{Numerical Integration {[}2 points{]}} Evaluate the three
distance decay functions along a line of around the central reference
point zero. \[
f_1(x) = exp(-x^1) \\
f_2(x) = exp(-x^2) \\
f_3(x) = exp(-x^3) 
\] over their full distance range \(0≤x≤∞\). Show the clean code for the
tasks below.

\hypertarget{task-5.1}{%
\subsection{Task 5.1}\label{task-5.1}}

Plot these three functions within a reasonable value range of the
distance x. How does the shape of the curves change with increasing
power?

\begin{Shaded}
\begin{Highlighting}[]
\NormalTok{x <-}\StringTok{ }\KeywordTok{seq}\NormalTok{(}\DataTypeTok{from =} \DecValTok{0}\NormalTok{,}\DataTypeTok{to =} \DecValTok{5}\NormalTok{,}\DataTypeTok{by =} \FloatTok{0.1}\NormalTok{)}
\NormalTok{y1 <-}\StringTok{ }\ControlFlowTok{function}\NormalTok{(x)\{}\KeywordTok{return}\NormalTok{(}\KeywordTok{exp}\NormalTok{(}\OperatorTok{-}\NormalTok{(x}\OperatorTok{^}\DecValTok{1}\NormalTok{)))\}}
\NormalTok{y2 <-}\StringTok{ }\ControlFlowTok{function}\NormalTok{(x)\{}\KeywordTok{return}\NormalTok{(}\KeywordTok{exp}\NormalTok{(}\OperatorTok{-}\NormalTok{(x}\OperatorTok{^}\DecValTok{2}\NormalTok{)))\}}
\NormalTok{y3 <-}\StringTok{ }\ControlFlowTok{function}\NormalTok{(x)\{}\KeywordTok{return}\NormalTok{(}\KeywordTok{exp}\NormalTok{(}\OperatorTok{-}\NormalTok{(x}\OperatorTok{^}\DecValTok{3}\NormalTok{)))\}}
\KeywordTok{plot}\NormalTok{(x,}\KeywordTok{y1}\NormalTok{(x),}\DataTypeTok{col =} \StringTok{'red'}\NormalTok{,}\DataTypeTok{type =} \StringTok{'l'}\NormalTok{)}
\KeywordTok{lines}\NormalTok{(x,}\KeywordTok{y2}\NormalTok{(x),}\DataTypeTok{col =} \StringTok{'blue'}\NormalTok{,}\DataTypeTok{type =} \StringTok{'l'}\NormalTok{)}
\KeywordTok{lines}\NormalTok{(x,}\KeywordTok{y3}\NormalTok{(x),}\DataTypeTok{col =} \StringTok{'green'}\NormalTok{,}\DataTypeTok{type =} \StringTok{'l'}\NormalTok{)}
\KeywordTok{legend}\NormalTok{(}\StringTok{"topright"}\NormalTok{, }\DataTypeTok{legend=}\KeywordTok{c}\NormalTok{(}\StringTok{"exp(-x^1)"}\NormalTok{, }\StringTok{"exp(-x^2)"}\NormalTok{,}\StringTok{"exp(-x^3)"}\NormalTok{),}
       \DataTypeTok{col=}\KeywordTok{c}\NormalTok{(}\StringTok{"red"}\NormalTok{, }\StringTok{"blue"}\NormalTok{,}\StringTok{'green'}\NormalTok{), }\DataTypeTok{lty=}\DecValTok{1}\NormalTok{, }\DataTypeTok{cex=}\FloatTok{0.8}\NormalTok{)}
\end{Highlighting}
\end{Shaded}

\includegraphics{GISC7310Lab01_files/figure-latex/unnamed-chunk-19-1.pdf}

Before \(x <1\), when the power goes up, the decrease speed of the
dependent variable (y) goes down. \(exp(-x^3 )>expa(-x^2 )>expa(-x^1 )\)
After \(x>1\), when the power goes up, the decrease speed of the
dependent variable (y) increases.

\hypertarget{task-5.2}{%
\subsection{Task 5.2}\label{task-5.2}}

Evaluate the areas \(A_1, A_2\) and \(A_3\) underneath the three curves
over their full support \(0≤x≤∞\) with \texttt{integrate(\ )} function.

\hypertarget{task-5.3}{%
\subsection{Task 5.3}\label{task-5.3}}

Calculate the expectation of the distances
\(E_i (x) = ∫_0^∞\frac{ x * (f_i (x))}{A_i} dx\) for all three distance
decay functions i∈\{1,2,3\}.

\begin{Shaded}
\begin{Highlighting}[]
\NormalTok{f <-}\StringTok{ }\ControlFlowTok{function}\NormalTok{(lambda)\{}
\NormalTok{  A <-}\StringTok{ }\KeywordTok{integrate}\NormalTok{(}\ControlFlowTok{function}\NormalTok{(x) \{}\KeywordTok{exp}\NormalTok{(}\OperatorTok{-}\NormalTok{x}\OperatorTok{^}\NormalTok{lambda)\}, }\DecValTok{0}\NormalTok{, }\OtherTok{Inf}\NormalTok{)}
\NormalTok{  fnE <-}\StringTok{ }\ControlFlowTok{function}\NormalTok{(x) \{x}\OperatorTok{*}\KeywordTok{exp}\NormalTok{(}\OperatorTok{-}\NormalTok{x}\OperatorTok{^}\NormalTok{lambda)}\OperatorTok{/}\NormalTok{A}\OperatorTok{$}\NormalTok{value\}}
\NormalTok{  E <-}\StringTok{ }\KeywordTok{integrate}\NormalTok{(fnE, }\DecValTok{0}\NormalTok{, }\OtherTok{Inf}\NormalTok{)}
\NormalTok{  fnV <-}\StringTok{ }\ControlFlowTok{function}\NormalTok{(x) \{((x}\OperatorTok{-}\NormalTok{E}\OperatorTok{$}\NormalTok{value)}\OperatorTok{^}\DecValTok{2}\NormalTok{)}\OperatorTok{*}\KeywordTok{exp}\NormalTok{(}\OperatorTok{-}\NormalTok{x}\OperatorTok{^}\NormalTok{lambda)}\OperatorTok{/}\NormalTok{A}\OperatorTok{$}\NormalTok{value\}}
\NormalTok{  Var <-}\StringTok{ }\KeywordTok{integrate}\NormalTok{(fnV, }\DecValTok{0}\NormalTok{, }\OtherTok{Inf}\NormalTok{)}
  \KeywordTok{return}\NormalTok{(}\KeywordTok{cbind}\NormalTok{(}\DataTypeTok{Lambda=}\NormalTok{lambda, }\DataTypeTok{Area=}\NormalTok{A}\OperatorTok{$}\NormalTok{value, }\DataTypeTok{Expectation=}\NormalTok{E}\OperatorTok{$}\NormalTok{value,}
               \DataTypeTok{Variance=}\NormalTok{Var}\OperatorTok{$}\NormalTok{value)) \}}
\end{Highlighting}
\end{Shaded}

\begin{Shaded}
\begin{Highlighting}[]
\KeywordTok{f}\NormalTok{(}\DecValTok{1}\NormalTok{)}
\end{Highlighting}
\end{Shaded}

\begin{verbatim}
##      Lambda Area Expectation Variance
## [1,]      1    1           1        1
\end{verbatim}

\begin{Shaded}
\begin{Highlighting}[]
\KeywordTok{f}\NormalTok{(}\DecValTok{2}\NormalTok{)}
\end{Highlighting}
\end{Shaded}

\begin{verbatim}
##      Lambda      Area Expectation  Variance
## [1,]      2 0.8862269   0.5641896 0.1816901
\end{verbatim}

\begin{Shaded}
\begin{Highlighting}[]
\KeywordTok{f}\NormalTok{(}\DecValTok{3}\NormalTok{)}
\end{Highlighting}
\end{Shaded}

\begin{verbatim}
##      Lambda      Area Expectation  Variance
## [1,]      3 0.8929795   0.5054681 0.1177842
\end{verbatim}

\hypertarget{task-6}{%
\section{Task 6}\label{task-6}}

\textbf{Appendix 1 {[}2 points{]}}

\hypertarget{task-6.1}{%
\subsection{Task 6.1}\label{task-6.1}}

Why is a t-distributed random variable 𝑇 with 𝑑𝑓𝑡 degrees of freedom
when it is squared 𝑇2 identically to the 𝐹-distributed random variable
with one degree of freedom in the numerator and 𝑑𝑓𝑡 degrees of freedom
for the denominator? Hint: use the definition of both random variables.
{[}0.8 points{]}

\end{document}
